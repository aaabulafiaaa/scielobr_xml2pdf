
%% bare_conf.tex
%% V1.3
%% 2007/01/11
%% by Michael Shell
%% See:
%% http://www.michaelshell.org/
%% for current contact information.
%%
%% This is a skeleton file demonstrating the use of IEEEtran.cls
%% (requires IEEEtran.cls version 1.7 or later) with an IEEE conference paper.
%%
%% Support sites:
%% http://www.michaelshell.org/tex/ieeetran/
%% http://www.ctan.org/tex-archive/macros/latex/contrib/IEEEtran/
%% and
%% http://www.ieee.org/

%%*************************************************************************
%% Legal Notice:
%% This code is offered as-is without any warranty either expressed or
%% implied; without even the implied warranty of MERCHANTABILITY or
%% FITNESS FOR A PARTICULAR PURPOSE! 
%% User assumes all risk.
%% In no event shall IEEE or any contributor to this code be liable for
%% any damages or losses, including, but not limited to, incidental,
%% consequential, or any other damages, resulting from the use or misuse
%% of any information contained here.
%%
%% All comments are the opinions of their respective authors and are not
%% necessarily endorsed by the IEEE.
%%
%% This work is distributed under the LaTeX Project Public License (LPPL)
%% ( http://www.latex-project.org/ ) version 1.3, and may be freely used,
%% distributed and modified. A copy of the LPPL, version 1.3, is included
%% in the base LaTeX documentation of all distributions of LaTeX released
%% 2003/12/01 or later.
%% Retain all contribution notices and credits.
%% ** Modified files should be clearly indicated as such, including  **
%% ** renaming them and changing author support contact information. **
%%
%% File list of work: IEEEtran.cls, IEEEtran_HOWTO.pdf, bare_adv.tex,
%%                    bare_conf.tex, bare_jrnl.tex, bare_jrnl_compsoc.tex
%%*************************************************************************

% *** Authors should verify (and, if needed, correct) their LaTeX system  ***
% *** with the testflow diagnostic prior to trusting their LaTeX platform ***
% *** with production work. IEEE's font choices can trigger bugs that do  ***
% *** not appear when using other class files.                            ***
% The testflow support page is at:
% http://www.michaelshell.org/tex/testflow/



% Note that the a4paper option is mainly intended so that authors in
% countries using A4 can easily print to A4 and see how their papers will
% look in print - the typesetting of the document will not typically be
% affected with changes in paper size (but the bottom and side margins will).
% Use the testflow package mentioned above to verify correct handling of
% both paper sizes by the user's LaTeX system.
%
% Also note that the "draftcls" or "draftclsnofoot", not "draft", option
% should be used if it is desired that the figures are to be displayed in
% draft mode.
%
\documentclass[conference]{IEEEtran}
% Add the compsoc option for Computer Society conferences.
%
% If IEEEtran.cls has not been installed into the LaTeX system files,
% manually specify the path to it like:
% \documentclass[conference]{../sty/IEEEtran}





% Some very useful LaTeX packages include:
% (uncomment the ones you want to load)


% *** MISC UTILITY PACKAGES ***
%
%\usepackage{ifpdf}
% Heiko Oberdiek's ifpdf.sty is very useful if you need conditional
% compilation based on whether the output is pdf or dvi.
% usage:
% \ifpdf
%   % pdf code
% \else
%   % dvi code
% \fi
% The latest version of ifpdf.sty can be obtained from:
% http://www.ctan.org/tex-archive/macros/latex/contrib/oberdiek/
% Also, note that IEEEtran.cls V1.7 and later provides a builtin
% \ifCLASSINFOpdf conditional that works the same way.
% When switching from latex to pdflatex and vice-versa, the compiler may
% have to be run twice to clear warning/error messages.






% *** CITATION PACKAGES ***
%
%\usepackage{cite}
% cite.sty was written by Donald Arseneau
% V1.6 and later of IEEEtran pre-defines the format of the cite.sty package
% \cite{} output to follow that of IEEE. Loading the cite package will
% result in citation numbers being automatically sorted and properly
% "compressed/ranged". e.g., [1], [9], [2], [7], [5], [6] without using
% cite.sty will become [1], [2], [5]--[7], [9] using cite.sty. cite.sty's
% \cite will automatically add leading space, if needed. Use cite.sty's
% noadjust option (cite.sty V3.8 and later) if you want to turn this off.
% cite.sty is already installed on most LaTeX systems. Be sure and use
% version 4.0 (2003-05-27) and later if using hyperref.sty. cite.sty does
% not currently provide for hyperlinked citations.
% The latest version can be obtained at:
% http://www.ctan.org/tex-archive/macros/latex/contrib/cite/
% The documentation is contained in the cite.sty file itself.






% *** GRAPHICS RELATED PACKAGES ***
%
\ifCLASSINFOpdf
  % \usepackage[pdftex]{graphicx}
  % declare the path(s) where your graphic files are
  % \graphicspath{{../pdf/}{../jpeg/}}
  % and their extensions so you won't have to specify these with
  % every instance of \includegraphics
  % \DeclareGraphicsExtensions{.pdf,.jpeg,.png}
\else
  % or other class option (dvipsone, dvipdf, if not using dvips). graphicx
  % will default to the driver specified in the system graphics.cfg if no
  % driver is specified.
  % \usepackage[dvips]{graphicx}
  % declare the path(s) where your graphic files are
  % \graphicspath{{../eps/}}
  % and their extensions so you won't have to specify these with
  % every instance of \includegraphics
  % \DeclareGraphicsExtensions{.eps}
\fi
% graphicx was written by David Carlisle and Sebastian Rahtz. It is
% required if you want graphics, photos, etc. graphicx.sty is already
% installed on most LaTeX systems. The latest version and documentation can
% be obtained at: 
% http://www.ctan.org/tex-archive/macros/latex/required/graphics/
% Another good source of documentation is "Using Imported Graphics in
% LaTeX2e" by Keith Reckdahl which can be found as epslatex.ps or
% epslatex.pdf at: http://www.ctan.org/tex-archive/info/
%
% latex, and pdflatex in dvi mode, support graphics in encapsulated
% postscript (.eps) format. pdflatex in pdf mode supports graphics
% in .pdf, .jpeg, .png and .mps (metapost) formats. Users should ensure
% that all non-photo figures use a vector format (.eps, .pdf, .mps) and
% not a bitmapped formats (.jpeg, .png). IEEE frowns on bitmapped formats
% which can result in "jaggedy"/blurry rendering of lines and letters as
% well as large increases in file sizes.
%
% You can find documentation about the pdfTeX application at:
% http://www.tug.org/applications/pdftex





% *** MATH PACKAGES ***
%
%\usepackage[cmex10]{amsmath}
% A popular package from the American Mathematical Society that provides
% many useful and powerful commands for dealing with mathematics. If using
% it, be sure to load this package with the cmex10 option to ensure that
% only type 1 fonts will utilized at all point sizes. Without this option,
% it is possible that some math symbols, particularly those within
% footnotes, will be rendered in bitmap form which will result in a
% document that can not be IEEE Xplore compliant!
%
% Also, note that the amsmath package sets \interdisplaylinepenalty to 10000
% thus preventing page breaks from occurring within multiline equations. Use:
%\interdisplaylinepenalty=2500
% after loading amsmath to restore such page breaks as IEEEtran.cls normally
% does. amsmath.sty is already installed on most LaTeX systems. The latest
% version and documentation can be obtained at:
% http://www.ctan.org/tex-archive/macros/latex/required/amslatex/math/





% *** SPECIALIZED LIST PACKAGES ***
%
%\usepackage{algorithmic}
% algorithmic.sty was written by Peter Williams and Rogerio Brito.
% This package provides an algorithmic environment fo describing algorithms.
% You can use the algorithmic environment in-text or within a figure
% environment to provide for a floating algorithm. Do NOT use the algorithm
% floating environment provided by algorithm.sty (by the same authors) or
% algorithm2e.sty (by Christophe Fiorio) as IEEE does not use dedicated
% algorithm float types and packages that provide these will not provide
% correct IEEE style captions. The latest version and documentation of
% algorithmic.sty can be obtained at:
% http://www.ctan.org/tex-archive/macros/latex/contrib/algorithms/
% There is also a support site at:
% http://algorithms.berlios.de/index.html
% Also of interest may be the (relatively newer and more customizable)
% algorithmicx.sty package by Szasz Janos:
% http://www.ctan.org/tex-archive/macros/latex/contrib/algorithmicx/




% *** ALIGNMENT PACKAGES ***
%
%\usepackage{array}
% Frank Mittelbach's and David Carlisle's array.sty patches and improves
% the standard LaTeX2e array and tabular environments to provide better
% appearance and additional user controls. As the default LaTeX2e table
% generation code is lacking to the point of almost being broken with
% respect to the quality of the end results, all users are strongly
% advised to use an enhanced (at the very least that provided by array.sty)
% set of table tools. array.sty is already installed on most systems. The
% latest version and documentation can be obtained at:
% http://www.ctan.org/tex-archive/macros/latex/required/tools/


%\usepackage{mdwmath}
%\usepackage{mdwtab}
% Also highly recommended is Mark Wooding's extremely powerful MDW tools,
% especially mdwmath.sty and mdwtab.sty which are used to format equations
% and tables, respectively. The MDWtools set is already installed on most
% LaTeX systems. The lastest version and documentation is available at:
% http://www.ctan.org/tex-archive/macros/latex/contrib/mdwtools/


% IEEEtran contains the IEEEeqnarray family of commands that can be used to
% generate multiline equations as well as matrices, tables, etc., of high
% quality.


%\usepackage{eqparbox}
% Also of notable interest is Scott Pakin's eqparbox package for creating
% (automatically sized) equal width boxes - aka "natural width parboxes".
% Available at:
% http://www.ctan.org/tex-archive/macros/latex/contrib/eqparbox/





% *** SUBFIGURE PACKAGES ***
%\usepackage[tight,footnotesize]{subfigure}
% subfigure.sty was written by Steven Douglas Cochran. This package makes it
% easy to put subfigures in your figures. e.g., "Figure 1a and 1b". For IEEE
% work, it is a good idea to load it with the tight package option to reduce
% the amount of white space around the subfigures. subfigure.sty is already
% installed on most LaTeX systems. The latest version and documentation can
% be obtained at:
% http://www.ctan.org/tex-archive/obsolete/macros/latex/contrib/subfigure/
% subfigure.sty has been superceeded by subfig.sty.



%\usepackage[caption=false]{caption}
%\usepackage[font=footnotesize]{subfig}
% subfig.sty, also written by Steven Douglas Cochran, is the modern
% replacement for subfigure.sty. However, subfig.sty requires and
% automatically loads Axel Sommerfeldt's caption.sty which will override
% IEEEtran.cls handling of captions and this will result in nonIEEE style
% figure/table captions. To prevent this problem, be sure and preload
% caption.sty with its "caption=false" package option. This is will preserve
% IEEEtran.cls handing of captions. Version 1.3 (2005/06/28) and later 
% (recommended due to many improvements over 1.2) of subfig.sty supports
% the caption=false option directly:
%\usepackage[caption=false,font=footnotesize]{subfig}
%
% The latest version and documentation can be obtained at:
% http://www.ctan.org/tex-archive/macros/latex/contrib/subfig/
% The latest version and documentation of caption.sty can be obtained at:
% http://www.ctan.org/tex-archive/macros/latex/contrib/caption/




% *** FLOAT PACKAGES ***
%
%\usepackage{fixltx2e}
% fixltx2e, the successor to the earlier fix2col.sty, was written by
% Frank Mittelbach and David Carlisle. This package corrects a few problems
% in the LaTeX2e kernel, the most notable of which is that in current
% LaTeX2e releases, the ordering of single and double column floats is not
% guaranteed to be preserved. Thus, an unpatched LaTeX2e can allow a
% single column figure to be placed prior to an earlier double column
% figure. The latest version and documentation can be found at:
% http://www.ctan.org/tex-archive/macros/latex/base/



%\usepackage{stfloats}
% stfloats.sty was written by Sigitas Tolusis. This package gives LaTeX2e
% the ability to do double column floats at the bottom of the page as well
% as the top. (e.g., "\begin{figure*}[!b]" is not normally possible in
% LaTeX2e). It also provides a command:
%\fnbelowfloat
% to enable the placement of footnotes below bottom floats (the standard
% LaTeX2e kernel puts them above bottom floats). This is an invasive package
% which rewrites many portions of the LaTeX2e float routines. It may not work
% with other packages that modify the LaTeX2e float routines. The latest
% version and documentation can be obtained at:
% http://www.ctan.org/tex-archive/macros/latex/contrib/sttools/
% Documentation is contained in the stfloats.sty comments as well as in the
% presfull.pdf file. Do not use the stfloats baselinefloat ability as IEEE
% does not allow \baselineskip to stretch. Authors submitting work to the
% IEEE should note that IEEE rarely uses double column equations and
% that authors should try to avoid such use. Do not be tempted to use the
% cuted.sty or midfloat.sty packages (also by Sigitas Tolusis) as IEEE does
% not format its papers in such ways.





% *** PDF, URL AND HYPERLINK PACKAGES ***
%
%\usepackage{url}
% url.sty was written by Donald Arseneau. It provides better support for
% handling and breaking URLs. url.sty is already installed on most LaTeX
% systems. The latest version can be obtained at:
% http://www.ctan.org/tex-archive/macros/latex/contrib/misc/
% Read the url.sty source comments for usage information. Basically,
% \url{my_url_here}.





% *** Do not adjust lengths that control margins, column widths, etc. ***
% *** Do not use packages that alter fonts (such as pslatex).         ***
% There should be no need to do such things with IEEEtran.cls V1.6 and later.
% (Unless specifically asked to do so by the journal or conference you plan
% to submit to, of course. )


% correct bad hyphenation here
\hyphenation{op-tical net-works semi-conduc-tor}


\begin{document}
%
% paper title
% can use linebreaks \\ within to get better formatting as desired

\title{Accelerating Curvatures Attributes on GPUs}


% author names and affiliations
% use a multiple column layout for up to three different
% affiliations
\author{\IEEEauthorblockN{Leonardo Martins}
\IEEEauthorblockA{Tecgraf/PUC-Rio\\
Rio de Janeiro\\
Email: lmartins@tecgraf.puc-rio.br}
\and
\IEEEauthorblockN{Marco Aurélio Gonçalves da Silva}
\IEEEauthorblockA{Tecgraf/PUC-Rio\\
Rio de Janeiro\\
Email: lmartins@tecgraf.puc-rio.br}
\and
\IEEEauthorblockN{Marcelo Arruda}
\IEEEauthorblockA{Tecgraf/PUC-Rio\\
Rio de Janeiro\\
Email: lmartins@tecgraf.puc-rio.br}
\and
\IEEEauthorblockN{Joner Jr Duarte}
\IEEEauthorblockA{Tecgraf/PUC-Rio\\
Rio de Janeiro\\
Email: lmartins@tecgraf.puc-rio.br}
\and
\IEEEauthorblockN{Paulo Souza}
\IEEEauthorblockA{Tecgraf/PUC-Rio\\
Rio de Janeiro\\
Email: lmartins@tecgraf.puc-rio.br}
\and
\IEEEauthorblockN{Pedro Mário}
\IEEEauthorblockA{Tecgraf/PUC-Rio\\
Rio de Janeiro\\
Email: lmartins@tecgraf.puc-rio.br}
\and
\IEEEauthorblockN{Roberto Beauclair}
\IEEEauthorblockA{Tecgraf/PUC-Rio\\
Rio de Janeiro\\
Email: lmartins@tecgraf.puc-rio.br}
\and
\IEEEauthorblockN{Marcelo Gattass}
\IEEEauthorblockA{Tecgraf/PUC-Rio\\
Rio de Janeiro\\
Email: lmartins@tecgraf.puc-rio.br}}

% conference papers do not typically use \thanks and this command
% is locked out in conference mode. If really needed, such as for
% the acknowledgment of grants, issue a \IEEEoverridecommandlockouts
% after \documentclass

% for over three affiliations, or if they all won't fit within the width
% of the page, use this alternative format:
% 
%\author{\IEEEauthorblockN{Michael Shell\IEEEauthorrefmark{1},
%Homer Simpson\IEEEauthorrefmark{2},
%James Kirk\IEEEauthorrefmark{3}, 
%Montgomery Scott\IEEEauthorrefmark{3} and
%Eldon Tyrell\IEEEauthorrefmark{4}}
%\IEEEauthorblockA{\IEEEauthorrefmark{1}School of Electrical and Computer Engineering\\
%Georgia Institute of Technology,
%Atlanta, Georgia 30332--0250\\ Email: see http://www.michaelshell.org/contact.html}
%\IEEEauthorblockA{\IEEEauthorrefmark{2}Twentieth Century Fox, Springfield, USA\\
%Email: homer@thesimpsons.com}
%\IEEEauthorblockA{\IEEEauthorrefmark{3}Starfleet Academy, San Francisco, California 96678-2391\\
%Telephone: (800) 555--1212, Fax: (888) 555--1212}
%\IEEEauthorblockA{\IEEEauthorrefmark{4}Tyrell Inc., 123 Replicant Street, Los Angeles, California 90210--4321}}




% use for special paper notices
%\IEEEspecialpapernotice{(Invited Paper)}




% make the title area
\maketitle


\begin{abstract}
%\boldmath
Curvature attributes have been widely used for visualization of folds, flexures, faults, among others interesting structural features in seismic data. This work presents a parallel approach to compute volumetric curvature attributes for seismic data on GPU and CPU (sequential and parallel). We compare the results using different derivative operator size for  sequential and the parallel approaches. The results show that for high derivative operator size the parallel approach performed on GPU can achieve further speed ups than the others approaches. Also, using the proposed GPU approach, it is possible to visualize volume sections in interaction time.
\end{abstract}

\IEEEpeerreviewmaketitle



\section{Introduction}
Volumetric curvature attributes have been featured as important tools for imaging and prediction of subtle fractures, faults, channels, among other structural and stratigraphic features. These seismic attributes provide, for each sample of the input data volume, important geometric information about the local surface around the sample. Thus, volumetric attributes of curvature may aid interpreters to delineate subtle stratigraphic features, as well as small-scale fracturing , and their analysis can be used as a powerful tool for structural and stratigraphic interpretation (Martins et al., 2012).

Although seismic attributes are useful for interpreters, its implementation can be very compute-intensive. For huge amount of input data, the time spent only on parameter testing may be prohibitive. Also, a continuous increasing of size in three-dimensional seismic data has been noticed in the past decades. Currently, in order to generate useful seismic attributes from these large amounts of data, a large amount of time may be required, even in high performance workstations. Hence, in seismic attribute research, the search for new technologies capable to accelerate processing speed, and consequently reduce processing time, is a great challenge.

Fortunately, since most seismic attributes algorithms are processed sample-by-sample, they are highly parallelizable. Guang-Zhi et al. (2012) use CUDA (Compute Unified Device Architecture) technology with an intent to accelerate instantaneous seismic attributes (envelop and its derivatives, phase, frequency and Q Factor) computation using GPU (Graphics Processing Unit). The GPU implementation make each thread responsible for process one seismic trace, so the parallel computing is realized among different seismic traces. The authors also compare the computational performance of CPU and GPU and show that GPU is about 6 times faster.

Jeong et al. (2006) present an interactive seismic fault detector that use a nonlinear set of filters implemented in GPU. The authors estimate local orientation through 3D structure tensors, which are used to guide an anisotropic diffusion filter. This filter reduces noise in the data while enhances fault discontinuities and reflectors' coherence. Next, a fault-likelihood volume is computed using a directional variance measure and fault samples are then extracted through a non-maximal-suppression method. Also, the authors show how the proposed algorithms are implemented with a GPU programming model, and a compare it with a CPU implementation.

Volumetric curvature attributes in seismic data were proposed by Al-Dossary and Marfurt (2006 ) . The authors demonstrate that volumetric curvature estimate can be obtained through dip volume derivatives. The use of fractional derivative operators allows a multi-spectral approach, highlighting features at different scales. In this approach, dip volumes are estimated through a local search for best correlation in semblance volume, and this is a very computationally costly step. Therefore, a larger search range can further increase the processing time.

A new method for volumetric curvature estimate was proposed by Martins et al. (2012). In this approach, local orientation is properly estimated taking into account the oscillatory aspect of amplitude seismic data. Thus, once we have a coherent normal field estimate, we can use differential geometry support in order to obtain several useful curvature attributes, such as maximum and minimum curvatures.

This work aims to present a direct comparison among sequential and parallel CPU implementations and CUDA implementation for curvature estimate in 3D seismic data. The method implemented in this paper is proposed by Martins et al. (2012). Multi-spectral curvature results can be obtained through the variation of derivative operator size. We compare processing time and speed-up gain for these implementations using several derivative operator sizes.

\section{Curvature}

The method implemented in this paper is based on three principal steps. First, seismic reflectors are properly described as level surfaces, through the computation of an horizon identifier attribute. In this work, we use vertical derivative for this purpose. Second, normal field along X, Y and Z directions is estimate based on the gradient of this attribute. Finally, normal field partial derivatives are calculated in order to find desired volumetric curvature attributes, using a differential geometry approach (Martins et al., 2012).

We noticed that the use of gaussian derivative operator provide a smoothed estimate of derivatives. Also, different derivative operator sizes increase or decrease curvature wavelength imaging (Al-Dossary and Marfurt, 2006). So, with an intent to generate low level detail results, it is necessary to increase filter size along the three directions, as well as high frequency results are obtained using small filter sizes.

\subsection{Parallel Approach}
For the method implemented in this work for volumetric curvature estimate, the main computational task is performed through convolution of gaussian derivative filters. Therefore, for parallelization purposes, this method has the advantage that a filter based on convolution can be straightly formulated in a parallel approach.

We use OpenMP (Open Multi-Processing) for CPU parallel implementation. OpenMP is an API that consists of a set of compiler directives, library routines, and environment variables that can be easily be added in source code to indicate parallel processing blocks. It supports multi-platform shared memory multiprocessing programming in C, C++, and Fortran. Through OpenMP, developers can balance processing effort among all available CPU cores.

GPU (Graphics Processing Unit) processing can be very helpful to parallelize these algorithms and make response time more feasible. Furthermore, its use is obtaining continuous acceptance in high performance processing field. In the past, GPU was used only in graphics applications. However, its high computational power soon aroused interest among researchers and propitiated its use in general-purposing computation. Also, due to the massive nature of seismic data, a massively parallel architecture could greatly benefit of the fact that these data are generally large in size.

In this context, CUDA (Compute Unified Device Architecture) is an useful technology that provide high level programming environments for GPU. Through CUDA, developers can access virtual instruction set and memory of GPU's directly, such that GPU's can be straightly used for general purpose parallel processing. Also, the CUDA programming language is very similar to C language.

Each of the three steps of the curvature method was implemented in a different CUDA kernel using blocks of 32x8 threads. To perform the convolution calculation of the three curvature steps, we could launch a thread for each sample of the target volume. That would be the straight implementation to the parallelization of the method. But after some optimization we realized that processing a line of samples in each thread showed up been a lot more efficient on the GPU due optimization of memory access.

\section{Results}
We executed the algorithm using different sizes of the derivative operator filter for the sequential approach running on the CPU and for the parallel approach both on CPU and GPU.

The resultant curvature for a 32 bits volume with 244MB (291 inlines, 476 crosslines and 462 time slices) is presented in Figure XXXX. The computational cost for a i7 3970x CPU (6 cores) with 24GB of RAM and a Geforce GTX 580 GPU (512 CUDA cores) is evaluated in Table 1.

\section{Conclusion}



As we can see in the results table, using a massively parallel architecture to calculate the curvature attribute can lead to a great improvement and can even make possible the process of one slice at interaction time, which allows the user to sweep the entire volume searching for a specific information helped by the curvature attribute.



\section*{Acknowledgment}


The authors would like to thank Petrobras for the research opportunity and for the permission to publish this work.



\begin{thebibliography}{1}

\bibitem{IEEEhowto:kopka}
H.~Kopka and P.~W. Daly, \emph{A Guide to \LaTeX}, 3rd~ed.\hskip 1em plus
  0.5em minus 0.4em\relax Harlow, England: Addison-Wesley, 1999.

\end{thebibliography}




% that's all folks
\end{document}


